\chapter{Conclusiones y posibles ampliaciones}

\section{Conclusiones}

Se consideran las siguientes conclusiones acerca de los objetivos propuestos:

\begin{itemize}

\item \textbf{Manejo de los logs}: Se ha conseguido una forma correcta de manejar la cantidad de logs generada dotándola de sentido para el observador gracias la utilización del ELK Stack. Kivana proporciona posibilidades de generación de informes con el fin de ayudar a los responsables a obtener información relevante a partir de los registros de la aplicación.

\item \textbf{Generación de alertas}: teniendo en cuenta que el proceso de detección de alertas se está ejecutando sobre un entorno con una cantidad de datos inferior a la real es lo suficiente rápido como para que suponga una mejora en los servicios de operación de la aplicación. El uso de ElastAlert es lo suficientemente flexible como para generar las alertas necesarias y las operaciones asociadas.
Para asegurar este comportamiento se realizarían pruebas de rendimiento sobre entornos de preproducción con cantidades de datos semejantes a producción.

\item \textbf{Obtención de anomalías}: la aplicación de LogReduce sobre los registros de errores desconocidos genera informes en los que efectivamente se puede observar la causa raíz del error. En este punto se obtienen falsos positivos debido a la estructura compleja de los logs y a que existen un número muy grande de posibilidades, sin embargo, aún no siendo un resultado óptimo si que se considera una mejora en el proceso de revisión.

\end{itemize}


\section{Posibles ampliaciones}

\begin{enumerate}
\item \textbf{Servidor Cloud ejecutando ELK}: la arquitectura propuesta de ELK podría estar alojada en un entorno Cloud como podría ser Amazon Web Services (AWS) o Windows Azure o bien servidores propios interconectados entre si. Tendía sentido dividir los distintos elementos en diferentes máquinas, por ejemplo, añadir un elemento como FileBeat de forma local, que envíe la información de los logs a un LogStash remoto y este, a su vez, enviara los datos refinados a otra máquina que ejecutaría Elastic Search y Kibana. Se podría garantizar la visualización de los datos a través de Kibana en una url fija accesible desde equipos remotos.
\item \textbf{Alertas por picos de actividad o tiempos de respuesta altos}: a la hora de monitorizar la actividad aportaría valor el hecho del registro de eventos que suponen un peor rendimiento para la aplicación. Podrían ser tiempos de respuesta altos que no lleguen a ser TimeOuts o registrar picos de usuarios a ciertas horas que podrían para localizar momentos en los que se espera que podría producirse caídas.
\item \textbf{Detector de anomalías}: se podría plantear añadir un módulo que se encargue de validar los comportamientos normales de la aplicación, con el fin de localizar y minimizar el fraude de los responsables que utilizar la aplicación. 
\end{enumerate}
