
%
%  PARA TRABALLOS EN GALLEGO USAR (LINEA 12): \usepackage[galician]{babel}
%  PARA TRABALLOS EN CASTELLANO USAR (LINEA 13): \usepackage[spanish]{babel}
%
% Para los acentos usamos codificacion UTF-8 (LINEA 10): \usepackage[utf8]{inputenc} 
% Si se usase la codificacion es_ES.ISO-8859-1 (LINEA 11): \usepackage[latin1]{inputenc}
% La conversion de acentos se hace con: iconv -f UTF-8 -t ISO-8859-1 filename.tex
%
% Como se incluyen figuras eps hay que compilar con: latex traballo , dvipdf traballo
%

\documentclass[12pt,twoside,a4paper,xcolor=table]{book}
% pódense engadir todos os packages necesarios
\usepackage[spanish]{babel}

\usepackage{multirow}
\usepackage[table,xcdraw]{xcolor}
\usepackage{graphicx}
\usepackage[dvips]{epsfig}
\usepackage{amssymb}
\usepackage{eurosym}
\usepackage{float}
\usepackage{latexsym}
\usepackage{a4}
\usepackage{xcolor,colortbl}
\usepackage{listings}
\usepackage{pdflscape}
\usepackage{tikz}
\usepackage{multirow, array} % para las tablas
\usepackage{float} % para usar [H]
% \usepackage{hyperref} % menús no pdf pero non leva ben co package galician



\addto\captionsgalician{\def\contentsname{Memoria tipo A -- \'{I}ndice xeral }}
	

\begin{document}
\pagestyle{empty}
\begin{center}
	{\bf\Large UNIVERSIDADE DE SANTIAGO DE COMPOSTELA}
	
	\vspace{0.5cm}
	\includegraphics[width=5cm]{figuras/logo_usc.eps}
	
	\vspace{0.5cm}
	{\bf\large ESCOLA TÉCNICA SUPERIOR DE ENXEÑARÍA}
	
	\vspace{2cm}
	{\bf\LARGE Sistema de alertas basado en procesamiento en tiempo real de logs en una plataforma con disponibilidad 24/7}
	
	%\vspace{0.5cm}
	%{\bf\LARGE Subtítulo do Traballo de Fin de Grao}
\end{center}

\vspace{2cm}
\hspace{4cm}\begin{tabular}{l}
	{\it\Large Autor/a:} \\
	{\bf\Large Adrián Bernárdez Cornes} \\
	~ \\
	{\it\Large Titores:} \\
	{\bf\Large Manuel Lama Penín} \\
\end{tabular}

\vspace{2cm}
\begin{center}
	{\bf\Large Máster universitario en Tecnologías de Análisis de Datos
Masivos: Big Data}
	
	\vspace{0.5cm}
	{\bf\large Julio 2022}
	
	\vspace{0.5cm}
	Traballo de Fin de Máster presentado na Escola Técnica Superior de Enxeñaría da Universidade de Santiago de Compostela para a obtención do Máster Interuniversitario en Big Data: Tecnologías de Análisis de Datos Masivos
\end{center}


\cleardoublepage

\cleardoublepage
\pagestyle{plain}
\tableofcontents

% Agora incluimos os capítulos. Cambiamos a numeración e as cabeceiras
\cleardoublepage
\pagenumbering{arabic}
\setcounter{page}{1}
\pagestyle{headings}
\setlength{\parskip}{8mm}
\chapter{Introdución}
\hyphenation{In-te-re-se}

Para un sistema de máxima disponibilidad como es el caso de sistemas de dispensación farmacéutica,
los cuales deben estar en funcionamiento las 24 horas del día, los 7 días de la semana se generan una
cantidad masiva de logs.

Para cumplir con esta disponibilidad se deberá disponer de un equipo de desarrollo y de soporte que cubrirán todas las horas del día haciendo guardias durante las horas no laborables. Sin embargo, en caso de que un componente falle, el tiempo de respuesta en informar del problema y buscar una solución deberá ser el mínimo posible independientemente de la causa y circunstancia en que se de el problema.

El objetivo del proyecto será aportar una plataforma que se encargue de generar alertas en caso de
una incidencia importante, analizando los logs de actividad hasta dar con el componente afectado. De esta forma el sistema deberá responder en tiempo real con el fin de agilizar el proceso y mejorar el servicio. Con esta nueva operativa se podría registrar estas alertas para poder dotar al equipo de un informe de las incidencias y sus componentes afectados, con el fin de elaborar soluciones a problemas comunes y poder formar de una manera más eficiente y exacta a los encargados del correcto funcionamiento del sistema principal.

Se buscará que la alerta generada sea lo más descriptible posible y que consiga dar un indicio de la causa
del problema. De esta manera se podrá dar una respuesta con tiempos mínimos tiempos de espera, lo
cual supondría tanto una mejora en el servicio como la facilitación del trabajo de cara al equipo
encargado de la supervisión de la aplicación.

A la hora realizar el diseño de la plataforma se deberá tener los siguientes puntos como claves: 

\begin{itemize}
\item Dada una aplicación que tenga una disponibilidad total y sobre la que opere un número elevado
de usuarios, se generan una cantidad de logs masiva, que en caso de que se produzca un error, el proceso de revisión de logs y la diagnosis posterior puede ocupar un tiempo
demasiado largo en caso de que no se pueden producir cortes en el servicio.

\item La finalidad del proyecto será disponer de una plataforma para detectar en tiempo real posibles problemas e incidencias y
clasificarlas según experiencias pasadas reducirá ese tiempo de espera en gran medida. La funcionalidad de la clasificación de la alerta entre problemas conocidos supondría una mejora significativa en la gestión operacional de la aplicación.

\item Registrar las incidencias alertadas supone un extra añadido, ya que se dispondrá de un histórico
con el que saldrán a la luz posibles bugs o errores frecuentes de cara a poder llevar un control
y análisis de la aplicación más exhaustiva.  

\item El proyecto deberá estar desacoplado de la aplicación principal y estar preparado para adaptarse a cambios y nuevas funcionalidades.

\end{itemize}



\section{Objetivos}

Una vez planteado el contexto se identifican los siguientes objetivos:

\definecolor{Gray}{gray}{0.85}
\newcolumntype{a}{&gt;{\columncolor{Gray}}c}
\newcolumntype{b}{&gt;{\columncolor{white}}c}

\begin{table}[H]
\centering
\renewcommand{\arraystretch}{1.5}
\begin{tabular}{| p{2.5cm} | p{10cm} |}
\hline
\bf{OBJ-01} \cellcolor{Gray} & \textbf{Emisión de alertas con información relevante tras la detección de posibles errores o
incidencias. Se buscará enviar la posible causa del problema raíz.} \\
\hline
\bf{Descripción} \cellcolor{Gray} & 
El principal objetivo del proyecto es la construcción de una plataforma con las herramientas necesarias para la emisión de alertas tempranas en caso de producirse una incidencia en la aplicación a monitorizar. Se espera que la alertas se generen en near real time con el fin de una recuperación del servicio los más rápida y eficiente posible. La plataforma deberá ser configurable ya que nace de la necesidad de coexistir servicio a un sistema con disponibilidad total que está en continuo desarrollo y evolución.
\\
\hline
\end{tabular}
\end{table}


\begin{table}[H]
\centering
\renewcommand{\arraystretch}{1.5}
\begin{tabular}{| p{2.5cm} | p{10cm} |}
\hline

\bf{OBJ-02} \cellcolor{Gray} & \textbf{Registro de las incidencias encontradas con la información necesaria para su posterior
tratamiento.} \\
\hline
\bf{Descripción} \cellcolor{Gray} & Se deberá llevar un registro de las incidencias detectadas y de los logs de la aplicación obtenidos. De esta forma una vez solucionada la incidencia se podrá catalogar y registrar los errores producidos en la aplicación con el fin de un análisis posterior en el que se podrán detectar posibles vulnerabilidades o bugs.
\\
\hline
\end{tabular}
\end{table}

\begin{table}[H]
\centering
\renewcommand{\arraystretch}{1.5}
\begin{tabular}{| p{2.5cm} | p{10cm} |}
\hline

\bf{OBJ-03} \cellcolor{Gray} & \textbf{Clasificar el error mediante los logs de la aplicación} \\
\hline
\bf{Descripción} \cellcolor{Gray} & La plataforma deberá utilizar técnicas de machine learning para la clasificación de la incidencia en base a los logs producidos en la aplicación. De esta forma se podrá plantear el envío de los pasos conocidos para su solución en cuanto se detecte el error.
\\
\hline
\end{tabular}
\end{table}


\section{Alcance}

La solución alcanzada deberá cumplir los objetivos definidos en el apartado anterior. Por un lado la plataforma a desarrollar deberá incorporar las herramientas necesarias que permitan de forma simple para un usuario la creación de alertas y modificación de las mismas en base a criterios acordados y a la experiencia obtenida. Por otro deberá utilizar los datos obtenidos con la finalidad de aplicar herramientas de Machine Learning con el fin de clasificar la incidencia ocurrida.

Se deberá tener en cuenta que no se podrán utilizar datos reales de la aplicación actual debido a que se estarían utilizando datos personales de terceros lo que incumpliría la normativa de protección de datos. En su lugar se utilizarán logs de entornos de desarrollo sobre los cuales se implementará una solución que permita crear un escenario equivalente al entorno real sin datos personales y a menor escala.

Una vez finalizado el desarrollo se deberá proporcionar una plataforma que pueda ser agregada a la aplicación actual y que deberá ser validada a posteriori. Para ello deberá proponerse una arquitectura como una extensión de la plataforma actual con el fin de que coexista. Por ello se ha de tener en cuenta que la instalación y despliegue será controlado en un entorno preparado para este fin permitiendo un control de versiones y registro del resultado.

Como entregables del proyecto serán los siguiente:
\begin{enumerate}
\item Plataforma encargada de la emisión de alertas con reglas de alertas configurables.
\item Plataforma encargada de la clasificación de las incidencias y su registro.
\item Propuesta de integración en el sistema real
\end{enumerate}
\clearpage
\chapter{Revisión de solucións comerciais}

\section{Gestión de alertas}

La monitorización de la actividad en la red puede ser un trabajo tedioso, sin embargo existen buenas razones pare hacerlo. Por un lado permite buscar e investigar inicios de sesión sospechosos en estaciones de trabajo, dispositivos conectados a redes y servidores mientras identifica las posibles deficiencia de seguridad en la adminitración del sistema. También puede reastrar las instalaciones software y las transferencias de datos para identificar problemas potenciales en tiempo real en lugar que después de que el daño esté hecho.

Estos logs trambién contribuyen en gran medida a que la organización cumpla con el reglamento general de protección de datos  (GHDPR) que se aplica a cualquier entidad que opere en la unión europea.

El registro, tanto el seguimiento como el análisis debe ser un proceso fundamental en cualquier infraestructura de monitoreo. Es necesario un archivo de registro de transacciones para recuperar una base de datos de un desastre. Además, al rastrear los archivos de registro, los equipo de DevOps y los administradores base de datos (DBA) pueden mantener un rendimiento óptimo de la base de datos o encontrar evidencia de actividad no autorizada en el caso de un ciberataque. Por esta razón, es importante monitorear y analizar regularmente los regitro del sistema. Es una forma confiable de recrear la cadena de eventos que condujo a cualquier problema que haya surgido.

Debido a esta serie de problemáticas, existen un número considerable de registro de código abierto y herramientas de análisis disponibles en la actualidad, lo que hace que elegir los recursos adecuados para los registros de actividad sea un trabajo considerable. La comunidad de software de código abierto y gratuito ofrece diseños de registros que funcionan con todo tipo de sitios y prácticamente con cualquier sistema operativo. 

Para la gestión de los logs y producción de alertas se tuvieron en cuenta las siguientes herramientas:

\subsection{Graylog}

Graylog comenzó en Alemania en 2011 y ahora se ofrece como una herramienta de código abierto o una solución comercial. Está diseñado para ser un sistema de administración de registros centralizado que recibe flujos de datos de varios servidores o puntos finales y le permite navegar o analizar esa información rápidamente.

Graylog se ha ganado una reputación positiva entre los administradores de sistemas debido a su facilidad de escalabilidad. La mayoría de los proyectos web comienzan con poco, pero pueden crecer exponencialmente. Graylog puede equilibrar cargas en una red de servidores backend y manejar varios terabytes de datos de registro cada día.

Los administradores de TI encontrarán que la interfaz frontend de Graylog es fácil de usar y robusta en su funcionalidad. Graylog se basa en el concepto de paneles, que le permite elegir qué métricas o fuentes de datos le parecen más valiosas y ver rápidamente las tendencias a lo largo del tiempo.

Cuando ocurre un incidente de seguridad o rendimiento, los administradores de TI quieren poder rastrear los síntomas hasta una causa raíz lo más rápido posible. La función de búsqueda en Graylog lo hace fácil. Tiene tolerancia a fallas incorporada que puede ejecutar búsquedas de múltiples subprocesos para que pueda analizar varias amenazas potenciales juntas.

\subsection{Nagios}

Nagios comenzó con un solo desarrollador en 1999 y desde entonces se ha convertido en una de las herramientas de código abierto más confiables para administrar datos de registro. La versión actual de Nagios puede integrarse con servidores que ejecutan Microsoft Windows, Linux o Unix.

Su producto principal es un servidor de registros, cuyo objetivo es simplificar la recopilación de datos y hacer que la información sea más accesible para los administradores del sistema. El motor del servidor de registro de Nagios capturará datos en tiempo real y los enviará a una poderosa herramienta de búsqueda. La integración con un nuevo punto final o aplicación es fácil gracias al asistente de configuración integrado.

Nagios se usa con mayor frecuencia en organizaciones que necesitan monitorear la seguridad de su red local. Puede auditar una variedad de eventos relacionados con la red y ayudar a automatizar la distribución de alertas. Nagios incluso se puede configurar para ejecutar scripts predefinidos si se cumple una determinada condición, lo que le permite resolver problemas antes de que un humano tenga que involucrarse.

Como parte de la auditoría de red, Nagios filtrará los datos de registro según la ubicación geográfica donde se originan. Eso significa que puede crear paneles de control completos con tecnología de mapeo para comprender cómo fluye su tráfico web.

\subsection{ELK Stack}

1806 / 5000
Resultados de traducción
Elastic Stack, a menudo llamado ELK Stack, es una de las herramientas de código abierto más populares entre las organizaciones que necesitan examinar grandes conjuntos de datos y dar sentido a los registros de su sistema.

Su oferta principal se compone de tres productos separados: Elasticsearch, Kibana y Logstash:

\begin{itemize}

\item Como sugiere su nombre, Elasticsearch está diseñado para ayudar a los usuarios a encontrar coincidencias dentro de conjuntos de datos utilizando una amplia gama de tipos y lenguajes de consulta. La velocidad es la ventaja número uno de esta herramienta. Se puede expandir en grupos de cientos de nodos de servidor para manejar petabytes de datos con facilidad.

\item Kibana es una herramienta de visualización que se ejecuta junto con Elasticsearch para permitir a los usuarios analizar sus datos y crear informes potentes. Cuando instale por primera vez el motor Kibana en su clúster de servidores, obtendrá acceso a una interfaz que muestra estadísticas, gráficos e incluso animaciones de sus datos.

\item La última pieza de ELK Stack es Logstash, que actúa como una canalización puramente del lado del servidor hacia la base de datos Elasticsearch. Puede integrar Logstash con una variedad de lenguajes de codificación y API para que la información de sus sitios web y aplicaciones móviles se alimente directamente a su potente motor de búsqueda Elastic Stalk.

\end{itemize}

Una característica única de ELK Stack es que le permite monitorear aplicaciones creadas en instalaciones de código abierto de WordPress. A diferencia de la mayoría de las herramientas de registro de auditoría de seguridad listas para usar que rastrean los registros de administración y PHP, pero poco más, ELK Stack puede examinar los registros del servidor web y de la base de datos.

El seguimiento de registros y la gestión de bases de datos deficientes son una de las causas más comunes del rendimiento deficiente de un sitio web. No verificar, optimizar y vaciar regularmente los registros de la base de datos no solo puede ralentizar un sitio, sino que también puede provocar un bloqueo completo. Por lo tanto, ELK Stack es una excelente herramienta para el kit de herramientas de todos los desarrolladores de WordPress.



\clearpage 
\chapter{Propuesta de solución}


Los archivos de registro de actividad que sobre los que se obtendrá la información del sistema se dividen en varios archivos, cada uno con un objetivo concreto. En concreto se monitorizarán tres ficheros, los cuales poseen un ciclo de vida que consiste en pasar por un fichero plano, para una vez alcanzado un tamaño determino comprimirse y guardarse como histórico. Si el histórico tiene una antigüedad superior a los días establecidos como máximo se enviarán a un servidor que funcionará como almacenamiento. Estos tres ficheros tienen los siguientes acometidos:

\begin{itemize}
\item Registro de interacciones (peticiones y respuestas) con otros sistemas externos e internos.
\item Registro de las transformaciones y cálculos que reciben los datos en los distintos servicios de los que se compone la aplicación.
\item Registro de la aplicación de entrada que utilizarán los usuarios finales con registro de las operaciones solicitadas y pantallas accedidas.
\end{itemize}

En los logs de actividad se registrarán el momento en que se registran, la clase en la que se originó el comportamiento y el texto con la información que se está registrando en ese momento. 

Estos ficheros se unificarán en un único índice de ElasticSearch por un motivo en concreto: todos los registros de actividad tienen asociado un identificador de operación, es decir, cuando se intenta hacer una operación sobre la aplicación los logs de actividad generados desde el origen hasta el final tendrán asociado un identificador numérico. De esta forma se podrá obtener rápidamente con una única búsqueda sobre ElasticSearch todos los registros de operación asociados a una operación en concreto.

\section{Arquitectura}

En este capítulo se va a describir la plataforma implementada para la emisión de alertas. La arquitectura diseñada para la plataforma es la siguiente:

\begin{figure}[H]
\centerline{\includegraphics[width=15cm]{figuras/arquitectura.png}}
\caption{Arquitectura de la Plataforma de Gestión de Alertas}
\label{enlace1}
\end{figure}

La arquitectura estará compuesta por distintos componentes desplegados cada uno en un contenedores Docker. Se pueden distinguir las siguientes partes:

\begin{itemize}
\item \textbf{Producer}: como se ha comentado en el capítulo de introducción no es posible utilizar los logs de la aplicación real en el entorno de producción ya que se estaría incumpliendo la ley de protección de datos. Para ello se utilizarán logs ya registrados de los que se tiene certeza que se produjo una incidencia pero procedente de un entorno de desarrollo con datos que no contengan información sensitiva. Para reproducir el comportamiento de la aplicación se implementará un productos python que se encargue de leer cada línea de log y trasladarla a kafka, de esta forma replicará el comportamiento de escribir registros a lo largo del tiempo. Esto productos deberá poder comportarse de forma distinta para poder realizar todas las ingestas de datos necesarias para la realización de pruebas.

\item \textbf{Kafka Container}: se utilizará kafka con el fin de trasladar los datos producidos en el productos al ELK Stack. La plataforma Apache Kafka es un sistema de transmisión de datos distribuido con capacidad de escalado y tolerante a fallos. Gracias a su alto rendimiento permitirá transmitir datos en tiempo real utilizando el patrón de mensajería publish/subscribe. Kafka se estructura logicamente en \textit{Topics}, que se puede definir como un flujo de datos sobre un tema en particular. Se podrán crear tantos como se desee y serán identificados por su nombre. Para cada \textit{Topic} existirán un número de particiones determinadas.

Cada elemento que se almacene en un \textit{Topic} se denomina \textit{Mensaje} o \textit{Record}, los cuales están compuestos por una clave y un valor, son inmutables y se añaden a una partición determinada dependiendo de la key (en caso de no especificar la key se aplica la estrategia "Round Robin" en la que cada mensaje es enviado a una partición distinta por orden). Cada mensaje dentro de una partición tiene un identificador numérico incremental llamado offset. Aunque los mensajes se guarden en los \textit{topics} por un tiempo limitado (una semana por defecto) y sean eliminados, el offset seguirá incrementando su valor. Los \textit{Brokers} son servidores identificados por un Id que contienen particiones de un \textit{topic}, no necesariamente todas. Kafka utiliza varios brokers para guardar copias de las particiones que actúan como maestro-esclavo y así ser tolerante a fallos. 

El contenedor docker estará compuesto por tres imágenes:

\begin{itemize}
\item \textbf{Kafka Cluster}: agrupación de N \textit{Brokers}.
\item \textbf{Zookeeper}: servicio centralizado encargado de gestionar los brokers de kafka manteniendo un listado con metadatos para las funciones de health checking. Se encarga de enviar las notificaciones en caso de cambios como la creación de nuevos topicos, caídas de brokes, etc.
\item \textbf{Kafka-manager}: interfaz gráfica que facilita la creación modificado y borrado de tópicos así como cambios en la configuración.
\end{itemize}

En el caso concreto del proyecto se utilizará unicamente un tópic que tendrá los datos leídos para los logs. Se desarrollará un productor usando \textit{Python} que se encargará de producir en el topic leyendo de los ficheros de log y como Consumidor de los datos se utilizará directamente Logstash, que funciona como un conector que lleva los datos a ElasticSearch.

Utilizando Kafka en este punto nos aporta las siguientes ventajas:

\begin{itemize}
\item En cualquier momento se podrá añadir nuevos Consumidores (fuentes de datos) y productores (nuevas salidas y tratamientos de los datos) de forma sencilla y sin cambios de configuración.
\item En caso de una caída del ELK Stack los datos se mantendrán en kafka por defecto una semana, por lo que no habría perdida de datos. Este tiempo máximo se puede configurar en cualquier momento utilizando la propiedad retention.ms y se podría alargar en caso de ser necesario.
\end{itemize}

\item \textbf{ELK Stack}: conjunto de aplicaciones (Elastic Search, Logstash y Kivana). Permite recoger datos de cualquier tpo de fuente y cualquier formato para realizar búsquedas, análisis y visualización de los datos en tiempo real. En la arquitectura planteada se para la aplicación real se obtendrán los datos a partir de ficheros de logs mediante Logstash y se almacenarán en el motor de búsquedas y análisis de Elasticsearch. Además, se permite la visualización, monitorización y explotación de datos en tiempo real mediante kibana. Estas tecnologías se explicaron con detalle en el punto \hyperref[sec:elk_stack]{\textit{ELK Stack}}.

\item \textbf{Generador de alertas}: este componente se encargará de la emisión de alertas utilizando para ello herramienta open source Elasticsearch, que se integra con ELK y permite detectar anomalías e inconsistencias en los datos. El conjunto de herramientas de \textbf{ELK (Elastic Stack)} más ElastAlert permite crear un sistema de monitorización para la extracción, almacenamiento y explotación de los datos de los procesos o sistemas a monitorizar. Además de esta herramienta se configurará en el mismo conector el proceso de reducción de logs, envío de emails y registro en base de datos.

\end{itemize}

En el sistema real se espera que las alertas se registren en la base de datos del sistema, sin embargo en el proyecto se utilizará un base de datos mongoDB desplegada en un contenedor Docker adicional.

\section{Generador de alertas}

Para detectar y emitir alertas como se comentó en el apartado anterior se utilizará ElastAlert.

ElastAlert es un componente originalmente diseñado por Yelp que es capaz de detectar anomalías, picos u otros patrones de interés. Se trata de un producto \textit{production-ready} y es un estándar de alerta conocido y aceptado dentro del ecosistema de ElasticSearch. En la documentación se especifica que "Si puede verlo en Kibana, ElastAlert puede alertarlo". ElastAlert podrá realizar distintas operaciones dependiendo que como se configuren las reglas en los ficheros de configuración. 

ElastAlert consulta periódicamente a ElasticSearch y los datos se trasladan al tipo de regla, la cual determina cuando se encuentra una coincidencia. Cuando se produce una coincidencia, se envía a una o más alertas que toman medidas en función de la coincidencia. Esto se configura mediante un conjunto de reglas, cada una de las cuales define una consulta, un tipo de regla y un conjunto de alertas. Se podrán definir tipos de regla del estilo:

\begin{itemize}
\item \textbf{Frequency}: existen una serie de X eventos en un tiempo Y. (Mismos códigos de error en un minuto)
\item \textbf{Spike}:la tasa de eventos aumenta/disminuye. (Número de peticiones por minuto sobre una operación concreta aumentan considerablemente) 
\item \textbf{Flatline}: menos de X eventos en Y tiempo. (No se realizan un mínimo de consultas a la bbdd en 1 minuto)
\item \textbf{Blacklist/Whitelist}: cuando un determinado campo coincide con una lista negra. (Se establecen unos códigos asociados a la indisponibilidad del sistema y en caso de detectarse se genera la alarma).
\item \textbf{Any}: cualquier evento que coincida con un filtro dado (Un código de error al que se le dará un trato especial).
\item \textbf{Change}: un campo tiene dos valores diferentes dentro de un tiempo (Una medida asociada a la temperatura varía drásticamente en un tiempo determinado).


\end{itemize}

En la solución propuesta se distinguirán dos tipos de alertas principales definidas tras un estudio de los logs.

\begin{itemize}
    \item \textbf{Errores conocidos}: en la aplicación se pueden dar ciertos errores relativos a problemas de comunicaciones con otros servicios, caídas de los propios servicios así como problemas de conexión o caída con las bases de datos. Cuando se produce un error de este tipo la mayoría de operaciones van a devolver en la respuesta un código de ERROR del cual se puede obtener la información necesaria para su diagnóstico.
    
    Para este tipo de errores se generara una lista de reglas con un comportamiento similar:
    \begin{enumerate}
        \item Se detectará que se está repitiendo un código de error un número configurable de veces en un tiempo determinado.
        \item Cada regla tendrá asignado un diagnóstico y las acciones a seguir en cada caso.
        \item Se enviará un correo especificando el error detectado, el diagnóstico y los pasos a seguir al correo de notificación de incidencias.
        
    \end{enumerate}
    
    \item \textbf{Errores desconocidos}: por otro lado pueden darse errores desconocidos en la aplicación. Estos al contrario que los anteriores no tienen un fácil diagnóstico ni hay una serie de acciones por defecto a seguir. Se puede tratar de un gran abanico de posibilidades a la hora de buscar la causa, desde un mal tratamiento de una respuesta de un sistema o un bug que se introdujo en el código. 
    
    En este tipo de errores se buscará realizar un tratamiento de los logs aplicando Machine Learning para detectar anomalías que puedan ayudar al responsable a diagnosticar y buscar una solución. En este caso se definirá un regla, que en caso de detectar un error no definido obtenga el identificador de la transacción de la operación y haga una llamada al servicio del componente de Alert Analist con el fin de que trate los logs obtenidos.
    
\end{itemize}

El punto más importante que hace que ElastAlert sea la herramienta escogida es la posibilidad de realizar acciones totalmente configurables una vez se active un trigger. Para el proyecto interesa el hecho de para una misma alerta poder enviar un correo informando del problema, el registro en base de datos y ejecutar el proceso de reducción de log en caso de que se trate de un error desconocido.


\subsubsection{Reducción de log}
\label{sec:log_reduction}

Cuando la aplicación falla con un error indeterminado se genera un log distinto al normal. Encontrar la causa del error puede ser un trabajo muy tedioso ya que involucra el estudio de los logs para encontrar la fracción del registro que se escapa de la operativa normal. Para hacer esta tarea más simple se utilizará LogReduce, el cual es un librería de código abierto que permite la generación de modelos Machine Learning entrenándolo con ejecuciones exitosas de procesos anteriores para extraer anomalías de los registros de ejecuciones fallidas.

Los logs de la aplicación se componen de un registro de los distintos servicios que permite la aplicación. Cada vez que un servicio es invocado se guardará un registro de la petición, las transformaciones realizadas sobre los datos, información referente a las casuísticas concretas y de la respuesta. El conjunto de los registros comunes en la ejecución de un servicio que finalice correctamente se conocerá como \textbf{\textit{Basenile}}.  En caso de se produzca algún tipo de error durante la ejecución del servicio se producirán excepciones que ayudarán a definir la causa del fallo.


Para eliminar los \textit{Baselines} de un log errado se podrá utilizar un algoritmo de k vecinos más próximos (\textbf{k-nearest neighbors pattern recognition algorithm k-NN}). 

Para ello, cada evento registrado en el log deberá ser convertido a un valor numérico aplicando algoritmos de Hashing Vectorizer. De esta forma se consigue mapear cada palabra y codificar cada evento en una matriz dispersa. Para mejorar este proceso se deberá aplicar una limpieza del texto quitando stop-words (palabras que no tienen significado por si solas), así como un proceso de tokenización para eliminar datos aleatorios como IPs y marcas temporales.

Una vez el modelo esté entrenado se podrá calcular la distancia de cada evento con respecto al \textit{Baseline} y mostrar aquellos resultados con un valor de distancia mayor que un umbral fijado. 

\begin{figure}[H]
\centerline{\includegraphics[width=15cm]{figuras/logreduce.png}}
\caption{Pipeline logreduce}
\label{enlace1}
\end{figure}

La librería LogReduce implementa este proceso con las ventajas de que se pueden especificar múltiples \textit{Baselines} para entrenar el modelo, así como la generación de reportes html para la mejor visualización del proceso.

El proceso de entrenamiento por tanto podrá ser cíclico de forma que en la que se vayan generando nuevos \textit{Baselines} añadirlos a la construcción del modelo. De esta forma el modelo estará preparado para cambios y evolucionará de igual manera que lo hace la aplicación. 


\chapter{Pruebas}

Con el fin de comprobar el correcto funcionamiento de la solución se realizarán distintas pruebas para corroborar que las alertas se generan correctamente y en un tiempo que pueda ser comprendido dentro de un sistema near real time. Para ello se obtendrá un conjunto de logs correctos y errores que se querrá identificar como una alerta (tanto de un error conocido con un error desconocido) con el objetivo de forzar una alerta y así comprobar que se recibe dentro del margen temporal esperado.

Debido a esto será necesario un trabajo previo de separación de logs en operaciones con errores y operaciones con funcionamiento correcto. Esto se llevará a cabo con un script Python, que separará en ficheros los dos tipos de operaciones a partir de logs producidos en un entorno no productivo de la aplicación. Una vez realizada la separación se podrá replicar el comportamiento de la aplicación a monitorizar, para lo que se desarrollarán dos scripts Python con las siguientes funciones:

\begin{itemize}
\item Script de carga inicial que permita una ingesta de datos hasta el tamaño deseado en Elastic Search. Para controlar el tamaño de datos cargados en elastic se podrá utilizar el comando:

\begin{verbatim}
curl 'elasticsearch:9200/_cat/indices?v' -H "Authorization: Basic ********************"
\end{verbatim}

\item Script que permita la publicación de logs en Kafka que reciba como entrada la cantidad de operaciones que se publican por segundo, el número de operaciones correctas que se publicarán hasta que se envíe el error y el tipo de error (conocido o desconocido).
\end{itemize}

Con esto conseguimos proporcionar el estado de alarma bajo demanda y así poder comprobar la respuesta en distintas situaciones. Por otro lado se desarrollará otro script que funcione como productor de la ingesta inicial de datos.

También interesa medir si se la información obtenida de aplicar la detección de anomalías sobre la traza de logs de errores desconocidos se considere útil. Para ello una vez obtenida la alerta en el informe generado deberá estar resaltado aquellas líneas de log que en un trabajo previo se han clasificado como las que contienen la causa de la incidencia.`

\subsection{Pruebas de Alertas y Rendimiento}

Para comprobar que las alertas aportan una ayuda y mejora al sistema real, se deberá probar que efectivamente se generan las alertas y se definirá un máximo de tiempo desde que se produce la incidencia hasta que el sistema la notifica. Se ha de tener en cuenta que se está trabajando sobre un prototipo que no corre sobre un sistema en producción. Debido a la menor capacidad de computo se generarán un número inferior de logs por segundo que el sistema real y se esperarán tiempos inferiores a 5 minutos entre el registro de la incidencia en los logs y la recepción del correo con la alerta. Se probarán distintos números de logs generados por segundo que sirva de estimación de comportamiento en el sistema real, así como distintos números de reglas de alertas desplegadas.

Con la realización de las pruebas se intentará conseguir tres objetivos:

\begin{itemize}


	\item \textbf{Ejecución correcta}: Se deberá comprobar que efectivamente se genera una alerta cuando se produce una incidencia en el sistema. Para que esto se considere correcto deberá generarse siempre una alerta en cada uno de los casos de prueba propuestos.

	\item \textbf{Obtención de la alerta en Near Real Time}: en caso de que el primer punto se valide correctamente, también se ha de comprobar el tiempo que transcurre desde que se genera la incidencia (en este caso mockeandola) hasta que la alerta es recibida es inferior al tiempo estimado.

	\item \textbf{Medida de recursos utilizados}: ya que los distintos componentes de la solución estarán desplegados en docker se deberá comprobar como afecta una escalada en el las cantidades de logs a analizar con el fin de estimar que recursos serían necesarios para poder tratar con el sistema real. Se tomará de medida el uso de CPU y memoria.

\end{itemize}

Para cada una de las pruebas con distintos número de reglas desplegadas se realizará el siguiente procedimiento:

\begin{enumerate}
	\item Despliegue de los contenedores Docker con la configuración y sin datos cargados.
	\item Para cada prueba individual se realiza la ingesta inicial de los datos   hasta el tamaño propuesto.
	\item Ejecución del script que genera la alarma con números distintos de reglas desplegadas.
	\item Registro de tiempos de respuesta y de recursos utilizados en los contenedores.
	\item Los puntos 4 y 5 se repetirá unas 5 veces y se calculará la media de los tiempos.
\end{enumerate}


Con el fin de comprobar como afecta el número de reglas que se tengan desplegadas se repetirán las pruebas con un número distinto y así establecer una aproximación de los recursos utilizados con varias pruebas. Además para facilitar el proceso durante las pruebas se establecerá que las alertas se ejecutan cuando se encuentre el primer error especificando con los parámetros:

\begin{verbatim}

# (Required, frequency specific)
# Alert when this many documents matching the query occur within a timeframe
num_events: 1

# (Required, frequency specific)
# num_events must occur within this amount of time to trigger an alert
timeframe:
    hours: 1
    
\end{verbatim}


\subsubsection{Muestra de los datos}

En la primera gráfica se mostrará la media de los tiempos obtenidos frente a los datos cargados en elastic search.

Como podemos observar un aumento substancial en la cantidad de datos de elastic search no supone un aumento significativo del tiempo de alerta. Tratándose de una POC en un entorno limitado el resultado obtenido supera con creces a lo esperado. Se puede esperar que en un entorno productivo el sistema funcione dentro de un rango de tiempo que suponga una mejora en el sistema de monitorizado. 

Por otro lado el consumo de recursos dependiendo del número de reglas es el siguiente:

Como se puede observar un aumento del número de reglas no implica un uso mas exhaustivo de los recursos. Incluso probando con un número de reglas mayor al esperando en entornos productivos el uso de memoria se puede considerar bajo.


\subsubsection{Conclusiones}

Una vez realizado el proceso de pruebas sobre el sistema se puede afirmar lo siguiente:

\begin{itemize}

\item El sistema cumple con la fiabilidad y tiempo esperado dentro de un entorno de pruebas acotado al hardware disponible.

\item Un aumento drástico en la cantidad de datos cargados en Elastic Search no supone una peor prestación en cuanto a tiempos de alertado.

\item El aumento de número de reglas no supone peores tiempos de alertado, aún que si que se puede percibir un pequeño mayor consumo de recursos. 

\end{itemize}


\subsection{Pruebas del detector de anomalías en errores desconocidos}

Como objetivo secundario de la aplicación se buscará probar que para los errores desconocidos se proporciona una herramienta que pueda ayudar a la resolución de la alerta.

Para esta prueba se han buscado operaciones fallidas con un error desconocido en las que se distinguieron claramente la causa del error. En este caso varias operaciones en las que se produjo una excepción java no controlada que llegó a registrarse.

Por otro lado, se realizará un entrenamiento inicial del modelo utilizando un conjunto logs de resultados correctos de la operación a probar. Para ello se utilizarán los logs utilizados en las pruebas de rendimiento y se filtrarán por operaciones para entrenar un modelo por operación. Una vez entrenados los modelos y aplicada la detección de anomalías se genera un informe que resalte aquellas líneas que difieran de las líneas base. 

Se definirá como correcto los casos en los que se diferencien aquellas líneas que en un trabajo previo se clasifiquen como las que aportan información para tratar la incidencia, es decir, las líneas que se consideren anómalas. El ejemplo de salida es el siguiente:

\begin{figure}[H]
\centerline{\includegraphics[width=15cm]{figuras/report.png}}
\caption{Ejemplo salida detección anomalías}
\label{enlace1}
\end{figure}


En las pruebas realizadas se han obtenido los siguientes resultados:

Positivos Correctos / Falsos Positivos

Negativos Correctos / Falsos Negativos


El resultado se considera correcto, ya que se efectivamente las líneas causantes del problema se resaltan como anomalías, sin embargo, no es un resultado perfecto, ya que también se están considerando anomalías (aunque con un valor de diferencia menor) debido al gran número de caminos y posibilidades posibles en las operaciones. Sin embargo, se considera que el resultado es lo suficientemente correcto, se espera que la salida pueda ayudar a personal no técnico encargado de la monitorización a facilitar el trabajo de revisión y búsqueda y con los resultados obtenidos se cumple esta condición.



%
% Engadir os capitulos que fagan falta
%
\clearpage
\chapter{Conclusiones y posibles ampliaciones}

\section{Conclusiones}

Se consideran las siguientes conclusiones acerca de los objetivos propuestos:

\begin{itemize}

\item \textbf{Manejo de los logs}: Se ha conseguido una forma correcta de manejar la cantidad de logs generada dotándola de sentido para el observador gracias la utilización del ELK Stack. Kivana proporciona posibilidades de generación de informes con el fin de ayudar a los responsables a obtener información relevante a partir de los registros de la aplicación.

\item \textbf{Generación de alertas}: teniendo en cuenta que el proceso de detección de alertas se está ejecutando sobre un entorno con una cantidad de datos inferior a la real es lo suficiente rápido como para que suponga una mejora en los servicios de operación de la aplicación. El uso de ElastAlert es lo suficientemente flexible como para generar las alertas necesarias y las operaciones asociadas.
Para asegurar este comportamiento se realizarían pruebas de rendimiento sobre entornos de preproducción con cantidades de datos semejantes a producción.

\item \textbf{Obtención de anomalías}: la aplicación de LogReduce sobre los registros de errores desconocidos genera informes en los que efectivamente se puede observar la causa raíz del error. En este punto se obtienen falsos positivos debido a la estructura compleja de los logs y a que existen un número muy grande de posibilidades, sin embargo, aún no siendo un resultado óptimo si que se considera una mejora en el proceso de revisión.

\end{itemize}


\section{Posibles ampliaciones}

\begin{enumerate}
\item \textbf{Servidor Cloud ejecutando ELK}: la arquitectura propuesta de ELK podría estar alojada en un entorno Cloud como podría ser Amazon Web Services (AWS) o Windows Azure o bien servidores propios interconectados entre si. Tendía sentido dividir los distintos elementos en diferentes máquinas, por ejemplo, añadir un elemento como FileBeat de forma local, que envíe la información de los logs a un LogStash remoto y este, a su vez, enviara los datos refinados a otra máquina que ejecutaría Elastic Search y Kibana. Se podría garantizar la visualización de los datos a través de Kibana en una url fija accesible desde equipos remotos.
\item \textbf{Alertas por picos de actividad o tiempos de respuesta altos}: a la hora de monitorizar la actividad aportaría valor el hecho del registro de eventos que suponen un peor rendimiento para la aplicación. Podrían ser tiempos de respuesta altos que no lleguen a ser TimeOuts o registrar picos de usuarios a ciertas horas que podrían para localizar momentos en los que se espera que podría producirse caídas.
\item \textbf{Detector de anomalías}: se podría plantear añadir un módulo que se encargue de validar los comportamientos normales de la aplicación, con el fin de localizar y minimizar el fraude de los responsables que utilizar la aplicación. 
\end{enumerate}


\clearpage
\markboth{BIBLIOGRAFÍA}{BIBLIOGRAFÍA}
\addcontentsline{toc}{chapter}{Bibliografía}


\begin{thebibliography}{99}
% EXEMPLO DE DOCUMENTO DESCARGADO DA WEB
\bibitem{logstash} Logstash Reference ({\it https://www.elastic.co/guide/en/logstash/current/index.html}). Consultado el 1 de octubre de 2021.

\bibitem{elastic.co}
E. BV, "elastic.co," Elasticsearch BV, 2021. [En línea]. Disponible:
https://www.elastic.co/guide/en/kibana/current/index.html. Consultado el 1 de octubre de 2021.

\bibitem{Libro 1}
Han, J. Kamber, M. Pei, J. Data Mining: Concepts and Techniques. Ed. 3,
Morgan Kaufmann Publishers Inc., 2003.



\end{thebibliography}



\end{document}
